\documentclass[11pt]{article}
\usepackage[a4paper,left=20mm,top=10mm]{geometry}
\usepackage[OT2]{fontenc}
\pagenumbering{gobble}
\usepackage{amsfonts}
\usepackage{amsthm}

\def \zn{,\kern-0.09em,}
\newcommand\eng{\fontencoding{OT1}\fontfamily{\rmdefault}\selectfont}
\newcommand\srb{\fontencoding{OT2}\fontfamily{\rmdefault}\selectfont}

\title{\bf{Moj dozhivljaj poeme \zn Oblak u pantalonama"}}
\author{\Large Aleksa Vuchkovic1, 3c}
\date{}

\begin{document}
\maketitle
\large Majakovski je bio buntovnik, protivnik izveshtachenih drushtvenih normi, chovek spreman na viku, dreku i proteste, ali i chovek chija je dusha bila satkana od chiste ljubavi, koja je zhivela za to da voli i bude voljena.\\

Pesma \zn Oblak u pantalonama" zapochinje prologom u kome se pesnik obrac1a svim onima koji su ga osudjivali. Govori im da ne znaju shta je ljubav, jer nisu u stanju da je iskuse i da se zhrtvuju. Prisec1a se svoje neuzvrac1ene, tragichne ljubavi i chekanja svoje voljene devojke jedne hladne decembarske vecheri. Pesnik je u tom trenutku vodio teshke unutrashnje borbe medju kojima je bilo i preispitivanje njegove ljubavi prema Mariji. Chekao je satima svoju dragu, da bi mu na kraju ona rekla da se udaje. Iako je njegovo lice izgledalo mirno, u njemu su se komeshala razna neprijatna osec1anja. To ga pogadja, ali joj ne dozvoljava da vidi njegovu bol, pa joj se obrac1a drsko i nepristojno. Govori joj da je glupa jer mu je davala lazhnu nadu, govorila je jedno, a radila neshto sasvim drugo. On proklinje drushtvo, Boga i Mariju shto su unishtili nadu u ljubav, medjutim pesnik svoje razocharenje pretvara u snagu koja mu pomazhe da nastavi dalje.\\

Vladimir Majakovski ovom pesmom prikazuje choveka koji je bio spreman da se suprot\/stavi primitivnoj sredini, koja ne razume njegove potrebe, zarad ljubavi koja ga je u isto vreme chinila besnim i ljutim, ali i nezhnim poput oblaka u pantalonama.

\end{document}