\documentclass{article}
\usepackage[ a4paper,total={170mm,257mm},left=20mm,top=0mm,bottom=10mm]{geometry}
\usepackage{multicol}
\usepackage{amssymb}
\usepackage{amsmath}
\newcommand\texteng{\fontencoding{OT1}\fontfamily{\rmdefault}\selectfont}
\pagenumbering{gobble}
\title{\textbf{\huge Geometric Inequalities}}
\date{}
\author{}
\begin{document}
\setlength{\columnsep}{1.2cm}
\maketitle
\begin{center}
\vspace*{-1.5cm}
 \large\textit{\textbf{Author:} Aleksa Vu\v ckovi\'c, first grade student, Mathematical Grammar School, Belgrade}\\
    \textit{\textbf{Supervisor:} Branko Grbi\'c}\\
    \normalsize\textit{Center for talented youth "Mihajlo Pupin", Dimitrija Tucovi\'ca 2, Pan\v cevo}
\end{center}
\begin{multicols}{2}
 \noindent
\section*{\underline{Introduction}}

\large A wide class of inequalities encountered in applications make geometric \text inequalities. Under geometric inequality, it is most commonly understood that inequality that \text applies to elements of a triangle or some other geometric figures (quadruple, couples, rollers, balls, etc.). In a wider sense, a geometric inequality is any inequality that \text relates to a geometric drawing.

\section*{\underline{Objective}}

The aim of the research work is to process and apply elemental inequalities in
continue mathematics in elementary and \text secondary school, as well as to prepare for
\text competition. The paper includes the synthesis of the \text theory with application, which is a standard of contemporary literature.

\section*{\underline{Materials and methods}}

Methods of research are collecting \text literature, as well as their interpretation and \text resolution.
The correlation between \text theoretical and practical considerations was sought
\text application, the process of \text reaching the solution is presented in detail. It's in work
the combined method of research used was studied
literature where there is a \text diverse application of inequality.
\noindent

\section*{\underline{Results}}

The object of this scientific research work is geometric inequalities in the plane, with a particular emphasis on inequalies in a \text triangle. In \text addition to \text elementary \text planimetric inequalities, Ptolemy's \text inequality $$AB\cdot CD+AD\cdot BC \geq AC\cdot BD.$$ was shown, which is valid in space, and whose applicability and effectiveness in \text solving the competition tasks is shown in this paper. In addition, Euler's \text inequality $$R\geq 2r$$ is also shown, which remains one of the greatest discoveries in this field of \text mathematics in the past two and a half \text centuries. Although simple, \text Euler's \text inequality is not trivial in any way and \text contributes to understanding the relationship between two important aspects of the triangle. The \text inequalities between means, which are showing the connection between the algebra and geometry itself in some way, are also \text presented in the picture. In addition to the aforementioned things, a selection of the most important things about this topic was made in the author's opinion with which the reader will meet in the upcoming paper.

\section*{\underline{Conclusion}}
Inequalities can be used
as a good \text manual to mathematicians, physicists, \text engineers,
\text mechanics, statisticians, economists. The application of inequality is present in \text mathematical analysis, geometry, \text probability theory, mathematical statistics, mathematical data processing, linear and dynamic programming, as well as in \text theoretical and applied mathematics.

\section*{\underline{Literature}}

[1] Z. Kadelburg, D. Djuki\'c, M. Luki\'c, I. Mati\'c, \textit{Nejednakosti}, second supplemented \text edition,  DMS, Beograd 2014.\\ {}
[2] M. Mitrovi\'c, S. Ognjanovi\'c, M. Veljkovi\'c, Lj. Petrovi\'c, N. Lazarevi\'c,\\ \textit{Geometrija za prvi razred Matemati\v cke \text gimnazije}, fourth edition, Krug, Belgrade 2013.\\{}
[3] Z. Cvetkovski,
\textit{\texteng{Inequalities, Theorems, Techniques and Selected Problems}}\\{} 
[4] {\texteng{O. Bottema}}, R. \v Z. Djordjevi\'c, R. R. Jani\'c, D. S. Mitrinovi\'c, P. M. Vasi\'c,\\
\texteng{\textit{Geometric Inequalities}, Wolters-Noordhoff Publishing, Groningen, Netherlands 1969.}

 \end{multicols}
  \pagebreak

\end{document}
