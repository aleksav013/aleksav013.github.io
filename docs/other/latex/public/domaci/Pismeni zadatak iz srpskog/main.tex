\documentclass[11pt]{article}
\usepackage[a4paper,left=20mm,top=10mm]{geometry}
\usepackage[OT2]{fontenc}
\pagenumbering{gobble}
\usepackage{amsfonts}
\usepackage{amsthm}

\title{\bf{Ovo prolec1e je drugachije}}
\author{\Large Aleksa Vuchkovic1, 2c}
\date{}

\begin{document}
\maketitle
\Large
Kao i svake godine, doshao je red i na prolec1e. Medjutim, ove godine se sve promenilo. Cvrkut ptica i veselu graju ljudi zamenila je monotona tishina. Od ovog trenutka, vishe nishta nec1e biti isto.

Sve je pochelo sa par glasina. Vrac1ao sam se iz shkole kao i svakog drugog dana. Nisam tada cenio slobodu koja mi je data, da uzhivam u ljudskom kontaktu i prelepoj prirodi koja me okruzhuje. U povratku kuc1i, imao sam priliku da saslusham nebrojeno mnogo ljudi koji su delili istu temu. Na vestima se jedino prichalo o tome. Odbijao sam da poverujem da je to stvarnost, ne s${}$hvatajuc1i ozbiljnost situacije. Ipak, stvari su se odvijale brzhe nego shto je iko mogao zamisliti. Ukinuta je shkola, uveden je policijski chas, vanredno stanje... Izgleda kao da vishe nishta nec1e biti isto. Ulice su postale prazne, chak je i priroda utihnula. Ljudi su se uplashili za svoje zdravlje i povukli su se u kuc1ice kao puzhevi. Znao sam da moram ostati kod kuc1e prvenstveno zbog mojih ukuc1ana, a potom i zbog sebe. Bilo mi je teshko u pochetku ali sam uspeo to da prevazidjem. Najtezhe mi je palo to shto vishe nisam mogao proshetam sa nekim, da ga zagrlim, porazgovaram, kao i da odolim tek razbudjenoj prirodi. Opojni cvetni pupoljci koji svojim mirisom privljache pazhnju svakoga ko uspe da ih namirishe iako josh nisu pokazali svoju pravu snagu, a tek olistalo lish${}$c1e ponosno krasi do tada ogoljene grane drvec1a. Nisam bio svestan koliko mi te sitnice znache sve dok ih nisam izgubio. Svi ti dosadni shkolski chasovi i naporan put do kuc1e bili su mi mnogo znachajniji nego shto sam to tada s${}$hvatao. Medjutim, iako sam privremeno ostao bez vazhnog aspekta u zhivotu, to me nije sprechilo da detaljno preispitam svoja osec1anja, i porazgovaram sa sobom. Spas od izolacije sam pronashao u boravku na selu i radu u bashti. Smatram da su ljudi na selu mozhda i najbolje proshli jer imaju svoj mali raj koji je samo njihov i dopushta im sve te stvari o kojima stanovnici gradova mogu samo da sanjaju.

Srec1om, polako se nazire kraj ove nepogode i stvari se vrac1aju u normalu. Ipak, ovo prolec1e c1e ostati upamc1eno kao prolec1e u doba korone.


\end{document}