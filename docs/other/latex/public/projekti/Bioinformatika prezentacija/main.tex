\documentclass[9pt]{beamer}
\mode<presentation>
\usetheme{Warsaw}

\definecolor{lblue}{RGB}{104, 216, 155}
\definecolor{mygreen}{rgb}{.125,.5,.25}
\usecolortheme[named=mygreen]{structure}

\definecolor{dgray}{RGB}{51, 51, 51}
\definecolor{lightgr}{rgb}{0.7 0.7 0.7}
\makeatletter
\addtobeamertemplate{footline}{%
  \color{lightgr}% to color the progressbar
  \hspace*{-\beamer@leftmargin}%
  \rule{\beamer@leftmargin}{2pt}%
  \rlap{\rule{\dimexpr
      \beamer@startpageofframe\dimexpr
      \beamer@rightmargin+\textwidth\relax/\beamer@endpageofdocument}{1.5pt}}
  % next 'empty' line is mandatory!

  \vspace{0\baselineskip}
  {}
}

\usepackage{listings}
%\usepackage{xcolor}
\lstset{
    language=C++,
    %basicstyle=\scriptsize,
    basicstyle=\fontsize{5}{5}\selectfont\ttfamily,
    keywordstyle=\color{mygreen},
    frame=single,
    showstringspaces=false,
    tabsize=4,
    escapeinside={<@}{@>},
}
\usepackage{hyperref}
\hypersetup{
    colorlinks,
    citecolor=black,
    filecolor=black,
    linkcolor=dgray,
    urlcolor=black
}
\usepackage[OT2]{fontenc}
\usepackage{amsfonts,amssymb,amsthm,amsmath}
\usepackage{listings}

\def\zn{,\kern-0.09em,}
\setlength\intextsep{0pt}
\renewcommand*\contentsname{Sadrzhaj}
\renewcommand*\refname{Literatura}
\newcommand\eng{\fontencoding{OT1}\fontfamily{\rmdefault}\selectfont}
\newcommand\srb{\fontencoding{OT2}\fontfamily{\rmdefault}\selectfont}

\title{\textbf{Bioinformatika}}
\author{Aleksa Vuchkovic1 3c}
\institute{\normalsize Matematichka gimnazija, Beograd}
\date{}

\begin{document}
\maketitle

\begin{frame}{Sadrzhaj}
\tableofcontents
\end{frame}
%%%%%%%%%%%%%%%%%%%%%%%%%%%%%%%%%%%%%%%%%%%%%%%%%%%%%%%%%%%%%%%%%%%%%%%%%%%%%%%%%%%%%%%%%%

\section{Uvod} % Sta je to bioinformatika
\begin{frame}{Uvod}
\begin{block}{}
Bioinformatika (grch. \eng{bios}\srb{} - zhivot; engl. \eng{informatics}\srb{}) je interdisciplinarna oblast koja razvija metode i alate za razumevanje bioloshkih podataka. Kao interdisciplinarno polje nauke, bioinformatika kombinuje informacione tehnologije, statistiku, matematiku i inzhinjerstvo kako bi analizirala i interpretirala bioloshke podatke. Bioinformatika se koristi u analizama simulacija bioloshkih pojava koristec1i matematichke i statistichke tehnike.
\end{block}
%Bioinformatika je zajednichki termin za oblast bioloshkih studija koje koriste kompjutersko programiranje kao deo svoje metodologije, i kao referenca za specifichne analize "toka podataka" koje se chesto koriste, posebno u podruchju genomike. Tipichna primena bioinformatike podrazumeva identifikaciju kandidata gena i nukleotida. Chesto je cilj njihove identifikacije bolje razumevanje genet\/ske osnove raznih bolesti, specifichnih prilagodjavanja organizama, zheljenih osobina (npr. u poljoprivrednim kulturama), ili razlika izmedju populacija. U manje formalnom tipu, bioinformatika takodje pokushava da otkrije organizacione principe unutar nukleinskih kiselina i proteinskih sekvenci.
\end{frame}
%%%%%%%%%%%%%%%%%%%%%%%%%%%%%%%%%%%%%%%%%%%%%%%%%%%%%%%%%%%%%%%%%%%%%%%%%%%%%%%%%%%%%%%%%%
\section{Cilj rada} %  Priblizhavanje pojma bioinformatike
\begin{frame}{Cilj rada}
\begin{block}{}
Cilj rada je bio obradjivanje osnovnih problema bioinformatike kroz reshavanje kako jednostavnijih tako i kompleksnijih i raznovrsnijih zadataka, kao i priblizhavanje pojma bioinformatike.
\end{block}
\onslide<2>
\begin{block}{}
Osim toga zheleo sam da pokazhem da sa znanjem programiranja i biologije stechenim u shkoli mozhemo da uchestvujemo u ovakvim projektima koji doprinose promociji nauke.
\end{block}
\end{frame}
%%%%%%%%%%%%%%%%%%%%%%%%%%%%%%%%%%%%%%%%%%%%%%%%%%%%%%%%%%%%%%%%%%%%%%%%%%%%%%%%%%%%%%%%%%
\section{Izrada zadataka u programskom jeziku \eng{C++}\srb} % Kod i objasnjenja
\begin{frame}{Izrada zadataka u programskom jeziku \eng{C++}\srb}
    
\begin{block}{}
Polazna tachka prilikom izbora tema su bili zadaci sa sajta \eng\url{https://rosalind.info}\srb. Rosalind je platforma za uchenje bioinformatike i programiranja kroz reshavanje zadataka. Svaki zadatak je struktuiran tako da korisnik dobije informaciju koju treba da obradi korish\/c1enjem programa kreiranog za tu specifichnu situaciju i na osnovu rezultata obrade se ispituje da li algoritam, kao i njegova implementacija predstavljaju reshenje zadatka.
\end{block}
\onslide<2>
\begin{block}{}
Svi programi koje c1ete videti u daljem tekstu su uspeshno proshli testiranje na sajtu. Osim toga uz reshenja zadataka dato je i objashnjenje teorije iz biologije potrebne za razumevanje zadatka.
\end{block}
%Bitno je napomenuti da sam kao autor tezhio da pishem pojednostavljen, kao i modularan kod tako da se neki delovi mogu koristiti vishe puta kao shto je na primer funkcija za chitanje koda u FASTA formatu koja sledi.
\end{frame}
%%%%%%%%%%%%%%%%%%%%%%%%%%%%%%
\subsection{Prebrojavanje azotnih baza polinukleotidnog lanca DNK}
\begin{frame}{Prebrojavanje azotnih baza polinukleotidnog lanca DNK}
\begin{block}{}
Za pochetak c1emo uraditi najjednostavniji zadatak kao uvod u deo rada sa kodom. Nash zadatak je da prebrojimo broj azotnih baza polinukleotidnog lanca DNK.
\end{block}
\onslide<2-3>
\begin{minipage}{\textwidth}\eng\lstinputlisting[]{cpp/counting.cpp}\srb\end{minipage}
\onslide<3>
\begin{minipage}{\textwidth}\eng\lstinputlisting[]{io/1io.txt}\srb\end{minipage}
\end{frame}
%%%%%%%%%%%%%%%%%%%%%%%%%%%%%%
\subsection{FASTA format}
\begin{frame}{FASTA format}
\begin{block}{}
U bioinformatici i biohemiji, FASTA format je format zasnovan na tekstu koji predstavlja nukleotidne sekvence ili sekvence aminokiselina (proteina), u kojima su nukleotidi ili aminokiseline predstavljeni pomoc1u jednoslovnih kodova. Format takodje omoguc1ava imenima sekvenci i komentarima da prethode sekvencama. Format potiche iz softverskog paketa FASTA, ali je sada postao gotovo univerzalni standard u oblasti bioinformatike.
\end{block}
\onslide<2>
\begin{block}{}
Ispod mozhete videti funkciju za chitanje fajla u FASTA formatu koju c1emo koristiti u svim preostalim zadacima koje obradjujemo u ovom radu.
\end{block}
\end{frame}
\begin{frame}
\begin{minipage}{\textwidth}\eng\lstinputlisting[]{cpp/fasta.cpp}\srb\end{minipage}
\onslide<2>
Primer korish\/c1enja te funkcije u programu:\\

\begin{minipage}{\textwidth}\eng\lstinputlisting[]{cpp/fasta_main.cpp}\srb\end{minipage}
\begin{minipage}{\textwidth}\eng\lstinputlisting[]{io/2io.txt}\srb\end{minipage}
\end{frame}
%%%%%%%%%%%%%%%%%%%%%%%%%%%%%%
\subsection{GC deo}
\begin{frame}{GC deo}
\begin{block}{}
U FASTA formatu dat nam je polinukleotidni lanac DNK za vishe subjekata i nash zadatak je da ispishemo ime subjekta sa najvec1im sadrzhajem GC(guanin-citozin) dela.
\end{block}
\onslide<2>
\begin{minipage}{\textwidth}\eng\lstinputlisting[]{cpp/gc_content.cpp}\srb\end{minipage}
\end{frame}
\begin{frame}{GC deo}
\begin{minipage}{\textwidth}\eng\lstinputlisting[]{io/3io.txt}\srb\end{minipage}
\end{frame}
%%%%%%%%%%%%%%%%%%%%%%%%%%%%%%
\subsection{Rachunanje mase proteina}
\begin{frame}{Rachunanje mase proteina}
\begin{block}{}
Dat nam je protein u obliku lanca aminokiselina obelezhenih engleskim alfabetom, kao i masa svake aminokiseline, a nash zadatak je da za dati lanac izrachunamo masu tog proteina.
\end{block}
\onslide<2-3>
\begin{block}{}
U strukturi podataka \eng{mapa }\srb c1mo chuvati podatke potrebne za izrachunavanje mase proteina na osnovu njihovog peptidnog lanca. Svaku aminokiselinu smo obelezhili sa velikim slovom engleske abecede(sva slova osim \eng{B,J,O,U,X,Z}\srb). U daljem tekstu mozhemo videti kako izgleda \eng{mapa }\srb za prvih 5 aminokiselina:
\end{block}
\onslide<3>
\begin{minipage}{\textwidth}\eng\lstinputlisting[linerange={1-5}]{cpp/protein_mass.h}\srb\end{minipage}
\end{frame}
\begin{frame}{Rachunanje mase proteina}
\begin{minipage}{\textwidth}\eng\lstinputlisting[]{cpp/protein_mass.cpp}\srb\end{minipage}
\onslide<2>
\begin{minipage}{\textwidth}\eng\lstinputlisting[]{io/4io.txt}\srb\end{minipage}
\end{frame}
%%%%%%%%%%%%%%%%%%%%%%%%%%%%%%
\subsection{Tranzicije i transverzije}
\begin{frame}{Tranzicije i transverzije}
\begin{block}{}
U ovom zadatku dat nam je polinukleotidni lanac pre i posle mutacija. Nash zadatak je da izrachunamo odnos broja tranzicija i broja transverzija. 
\end{block}
\onslide<2>
\begin{block}{}
Tranzicije su tip tachkaste mutacije kada se nukleotidna baza menja iz jedne purinske baze u drugu ($\textrm{A} \leftrightarrow \textrm{G}$) ili iz jedne pirimidinske u drugu ($\textrm{C} \leftrightarrow \textrm{T}$), a transverzije su kada se nukleotidna baza menja iz pirimidinske u purinsku bazu i obrnuto. U ovom zadatku tachkaste mutacije su zamena jedne nukletotidne baze. Ovaj odnos nam daje brzu i korisnu statistiku za analizu genoma.
\end{block}
\end{frame}
\begin{frame}{Tranzicije i transverzije}
\begin{minipage}{\textwidth}\eng\lstinputlisting[]{cpp/transitions_and_transversions.cpp}\srb\end{minipage}
\onslide<2>
\begin{minipage}{\textwidth}\eng\lstinputlisting[]{io/5io.txt}\srb\end{minipage}
\end{frame}
%%%%%%%%%%%%%%%%%%%%%%%%%%%%%%
\subsection{Splajsovanje RNK/Eksoni i introni}
\begin{frame}{Splajsovanje RNK/Eksoni i introni}
Pre nego shto predjemo na zahtev zadatka moramo videti shta je splajsovanje:
\onslide<2-3>
\begin{block}{}
U molekularnoj biologiji i genetici, \textbf{splajsovanje} je modifikacija RNK nakon transkripcije, u kojoj se introni uklanjaju, a eksoni se spajaju. Ono je neophodno da bi tipichna eukariotska informaciona RNK mogla da se koristi za proizvodjenje korektnog proteina putem translacije. 
\end{block}
\onslide<3>
\begin{block}{}
U ovom zadatku dat nam je polinukleotidni DNK lanac, kao i niz lanaca koji predstavljaju introne. Cilj nam je da ispishemo kako bi izgledao peptidni niz za dati DNK lanac.
\end{block}
\end{frame}
\begin{frame}{Splajsovanje RNK/Eksoni i introni}
\begin{block}{}
U strukturi podataka \eng{mapa }\srb c1emo chuvati podatke potrebne za translaciju(sintezu proteina). Svaki kodon se translira u neku od aminokiselina koju smo obelezhili sa velikim slovima engleske abecede(sva slova osim \eng{B,J,O,U,X,Z}\srb). U daljem tekstu mozhemo videti kako izgleda \eng{mapa }\srb za prvih 10 kodona:
\end{block}
\onslide<2>
\begin{minipage}{\textwidth}\eng\lstinputlisting[linerange={1-10}]{cpp/rna_splicing.h}\srb\end{minipage}
\end{frame}
\begin{frame}{Splajsovanje RNK/Eksoni i introni}
\begin{minipage}{\textwidth}\eng\lstinputlisting[]{cpp/rna_splicing.cpp}\srb\end{minipage}
\onslide<2>
\begin{minipage}{\textwidth}\eng\lstinputlisting[]{io/6io.txt}\srb\end{minipage}
\end{frame}
%%%%%%%%%%%%%%%%%%%%%%%%%%%%%%%%%%%%%%%%%%%%%%%%%%%%%%%%%%%%%%%%%%%%%%%%%%%%%%%%%%%%%%%%%%
\section{Zakljuchak}
\begin{frame}{Zakljuchak}
\begin{block}{}
Informatika je postala deo svega pa tako i biologije. Olakshala je istrazhivanja i doprinela formiranju biologije kao nauke. Mnogi bioloshki i biohemijski problemi mogu se efikasno reshiti programskom implementacijom odgovarajuc1ih algoritama. S obzirom da zhivimo u vremenu kada tehnologija brzo napreduje, korisno je fokusirati se na trazhenju shto vishe informatichkih reshenja problema iz ovih oblasti. Zbog toga mislim da je bitno raditi na razvoju bioinformatike, kao jedne od najnaprednijih i najznachajnijih bioloshkih disciplina.
\end{block}
\end{frame}
%%%%%%%%%%%%%%%%%%%%%%%%%%%%%%%%%%%%%%%%%%%%%%%%%%%%%%%%%%%%%%%%%%%%%%%%%%%%%%%%%%%%%%%%%%
\section{Literatura}
\begin{frame}{Literatura}
\begin{thebibliography}{}
\bibitem{zad} Zadaci korish\/c1eni u radu \eng\url{http://rosalind.info/problems/list-view/}\srb
\onslide<2-3>
\bibitem{molbio} Jelena Popovic1, \textit{Molekularna biologija za 4. razred Matematichke Gimnazije}, \eng\url{https://www.mg.edu.rs/uploads/files/images/stories/dokumenta/profesori/jelena-popovic/molekularna-biologija.docx}\srb
\onslide<3>
\bibitem{masa} Tablica monoizotopnih masa aminokiselina \eng\url{http://rosalind.info/glossary/monoisotopic-mass-table/}\srb
\bibitem{tran} Tablica translacije kodona u aminokiseline \eng\url{http://rosalind.info/glossary/rna-codon-table/}\srb
\bibitem{rec} Rechnik pojmova i redje korish\/c1enih izraza \eng\url{http://rosalind.info/glossary}\srb
\end{thebibliography}
\end{frame}
\section{}
\begin{frame}
\centering\LARGE HVALA NA PAZHNJI!
\end{frame}
\end{document}