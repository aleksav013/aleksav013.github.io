\documentclass{beamer}

\mode<presentation>
\usetheme{Warsaw}
%\setbeamercolor{uppercol}{fg=white,bg=green!75!blue}

\usepackage{standalone}
\usepackage{tikz}
\usepackage{tikz-3dplot}
\usepackage{fouriernc}
\usepackage{tkz-euclide}
\usetkzobj{all}
\usetikzlibrary{calc,backgrounds}

\definecolor{mygreen}{rgb}{.125,.5,.25}
\usecolortheme[named=mygreen]{structure}

\definecolor{lightgr}{rgb}{0.7 0.7 0.7}
\makeatletter
\addtobeamertemplate{footline}{%
  \color{lightgr}% to color the progressbar
  \hspace*{-\beamer@leftmargin}%
  \rule{\beamer@leftmargin}{2pt}%
  \rlap{\rule{\dimexpr
      \beamer@startpageofframe\dimexpr
      \beamer@rightmargin+\textwidth\relax/\beamer@endpageofdocument}{1.5pt}}
  % next 'empty' line is mandatory!

  \vspace{0\baselineskip}
  {}
}

\usepackage[OT2]{fontenc}
\usepackage{amsfonts,amssymb,amsthm,amsmath}
\usepackage{graphicx}
\usepackage{wrapfig}
\usepackage{subcaption}
\usepackage{multicol}

\def \zn{,\kern-0.09em,}
\setlength\intextsep{0pt}

\newcommand\eng{\fontencoding{OT1}\fontfamily{\rmdefault}\selectfont}
\def\ug{\mathbin{\sphericalangle\,}}
\def\dj{d\kern-0.4em\char"16\kern-0.1em}
\def\Dj{\mbox{\raise0.3ex\hbox{-}\kern-0.4em D}}
\newcommand{\D}{\displaystyle}

\title{\textbf{Upisana i opisana lopta}}
\author{Aleksa Vuchkovic1}
\date{}
\institute{Matematichka gimnazija, Beograd}
\logo{\includegraphics[scale=0.08]{logo.png}}



\begin{document}
\maketitle

\begin{frame}{Sadrzhaj}
\tableofcontents
\end{frame}

\section{Uvod}
\begin{frame}{Uvod}
\begin{block}{}
Lopta je geometrijsko telo ogranicheno sferom. Chine je sve tachke koje su manje ili jednako udaljene $r$ od centra.
\end{block}
\begin{multicols}{2}
\onslide <2-3>
\begin{figure}[H]
    \includestandalone[scale=0.7]{lopta}
\end{figure}

\columnbreak
\onslide <3>
\leavevmode\\[10pt]
\Large$P=4r^2\pi$\\
\Large$V=\frac{4}{3}r^3\pi$

\end{multicols}
\end{frame}

\section{Opisana lopta}
\subsection{...oko valjka}
\begin{frame}
\begin{block}{}
Lopta je opisana oko valjka ako su osnove valjka preseci lopte. Oko svakog
pravog valjka mozhe se opisati lopta.
\end{block}
\onslide<2>\begin{figure}[H]
    \includestandalone[scale=0.7]{opisana/valjak}
\end{figure}
\end{frame}
\subsection{...oko prizme}
\begin{frame}
\begin{block}{}
Da bi se oko prizme mogla opisati sfera potrebno je i dovoljno da prizma bude
prava i da se oko njene osnove mozhe opisati krug.
\end{block}
\onslide<2>\begin{figure}[H]
    \includestandalone[scale=0.7]{opisana/prizma}
\end{figure}
\end{frame}
\subsection{...oko piramide}
\begin{frame}
\begin{block}{}
Da bi se oko piramide mogla opisati sfera potrebno je i dovoljno da se oko
njene osnove mozhe opisati krug.
\end{block}
\onslide<2>\begin{figure}[H]
    \includestandalone[scale=0.7]{opisana/piramida}
\end{figure}
\end{frame}
\subsection{...oko poliedra}
\subsection{...oko kupe}
\begin{frame}
\begin{block}{}
Lopta je opisana oko kupe ako je osnova kupe presek lopte i ako vrh kupe
pripada odgovarajuc1oj sferi. Oko svake kupe mozhe se opisati lopta.
\end{block}
\onslide<2>\begin{figure}[H]
    \includestandalone[scale=1]{opisana/kupa}
\end{figure}
\end{frame}
\section{Upisana lopta}
\subsection{...u valjak}
\begin{frame}
\begin{block}{}
Lopta je upisana u prav valjak ako osnove i sve izvodnice valjka dodiruju
loptu. To je moguc1e ako je prechnik osnove valjka jednak visini valjka.
\end{block}
\onslide<2>\begin{figure}[H]
    \includestandalone[scale=0.8]{upisana/valjak}
\end{figure}
\end{frame}
\subsection{...u prizmu}
\begin{frame}
\begin{block}{}
Da bi se u prizmu mogla upisati sfera potrebno je i dovoljno da se u njen normalni presek mozhe upisati krug chiji je prechnik jednak visini prizme. 
\end{block}
\onslide<2>\begin{figure}[H]
    \includestandalone[scale=0.7]{upisana/prizma}
\end{figure}
\end{frame}
\subsection{...u piramidu}
\begin{frame}
\begin{block}{}
Da bi se u piramidu mogla upisati sfera dovoljno je da nagibni uglovi bochnih
strana prema osnovi piramide budu jednaki.
\end{block}
\onslide<2>\begin{figure}[H]
    \includestandalone[scale=0.7]{upisana/piramida}
\end{figure}
\end{frame}
\subsection{...u kupu}
\begin{frame}
\begin{block}{}
Lopta je upisana u pravu kupu ako osnova i sve izvodnice kupe dodiruju
loptu. To je uvek moguc1e!  
\end{block}
\onslide<2>\begin{figure}[H]
    \includestandalone[scale=1]{upisana/kupa}
\end{figure}
\end{frame}
\section{Zadaci}
\begin{frame}{Zadatak 1.}
\begin{block}{$VI$ Rotaciona tela, 70. str./82. zadatak iz ud2benika}
Kupa visine $H$ je upisana u loptu. Nac1i zapreminu lopte ako je njena zapremina 4 puta vec1a od zapremine kupe.
\end{block}
\begin{multicols}{2}
\onslide <2-4>
\begin{figure}[H]
    \includestandalone[scale=0.8]{zadaci/1zad}
\end{figure}

\columnbreak
\onslide <3-4>
$R^2=r^2+(H-R)^2$\\
$\frac{4}{3}R^3\cdot\pi=4\cdot \frac{1}{3}r^2\pi H$\\
$R^3=r^2\cdot H$\\
\onslide <4>
$R^2=\frac{R^3}{H}+(H-R)^2/\cdot H$\\
$R^3-2RH^2+H^3=0$\\
$(R-H)\cdot(R^2-HR+H^2)=0$ \\
$\Rightarrow R=H$
\end{multicols}
\end{frame}
\begin{frame}{Zadatak 2.}
\begin{block}{$VI$ Rotaciona tela, 70. str./76. zadatak iz ud2benika}
Oko prave jednakoivichne trostrane prizme ivice $a$ opisana je lopta. Koliki je njen poluprechnik?
\end{block}
\begin{multicols}{2}
\onslide <2-3>
\begin{figure}[H]
    \includestandalone[scale=0.5]{zadaci/2zad}
\end{figure}

\columnbreak
\onslide <3>
$R^2=\left(\frac{a}{2}\right)^2+\left(\frac{a\sqrt{3}}{3}\right)^2$\\
$R=\frac{a\sqrt{21}}{6}$
\end{multicols}
\end{frame}
\begin{frame}{Zadatak 3.}
  \begin{block}{$VI$ Rotaciona tela, 70. str./98. zadatak iz ud2benika}
U jednakoivichnoj  chetvorostranoj piramidi upisana je sfera poluprechnika $r=\frac{\sqrt{2}}{1+\sqrt{3}}$. Izrachunati zapreminu te piramide.
\end{block}
\begin{multicols}{2}
\onslide <2-3>
\begin{figure}[H]
    \includestandalone[scale=0.5]{zadaci/3zad}
\end{figure}

\columnbreak
\onslide <3>
$a=2$\\
$H=\sqrt{2}$\\[2mm]
$V=\frac{4\sqrt{2}}{3}$
\end{multicols} 
\end{frame}
\section{}
\begin{frame}
\centering\LARGE HVALA NA PAZHNJI!
\end{frame}
\end{document}