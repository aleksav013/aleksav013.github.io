\documentclass[11pt]{article}
\usepackage[ a4paper,total={170mm,257mm},left=20mm,top=10mm]{geometry}
\usepackage[OT2]{fontenc}
\pagenumbering{gobble}
\usepackage{amsthm}

\newcommand\tsd{\theoremstyle{definition}}
\newcommand\tsr{\theoremstyle{remark}}
\setlength{\parindent}{0em}
\def \zn{,\kern-0.09em,}

\tsd\newtheorem{p}{}
\newcommand\eng{\fontencoding{OT1}\fontfamily{\rmdefault}\selectfont}
\newcommand\srb{\fontencoding{OT2}\fontfamily{\rmdefault}\selectfont}

\title{\bf{\zn Chicha Gorio", Onore de Balzak}}
\author{\Large Aleksa Vuchkovic1 2c}
\date{}

\begin{document}
\maketitle
\large
\begin{p}
\textbf{Onore de Balzak - kratka biografija}\\
Onore de Balzak rodjen je u gradu Tur, 20. maja 1799., a preminuo je u Parizu, 18. avgusta 1850. Bio je francuski romanopisac koji se smatra kljuchnim autorom realizma. Zajednichki naziv za njegova dela je \zn Ljudska komedija" koja je prvobitno planirana sa 137 dela, ali je od tog broja napisano \zn samo" 91 delo. Neka od najznachajnijih dela su i \zn Evgenije Grande", kao i \zn Shagrinska kozha".
\end{p}

\begin{p}
\textbf{Izvrshite lokalizaciju teksta - mesto i vreme.}\\
Radnja u romanu se odvija u periodu od kraja novembra 1819. do 21. februara 1820. godine u gradu Parizu u Francuskoj. Najvec1i broj dogadjaja uochava se u pansionu gospodje Voke u ulici Nev-Sent-Zhernejev. Manje vazhne su i kuc1a viskontese de Bosean, kao i Jelisejska polja - avenija koja vodi ka svim pozorishtima i luksuznim radnjama, jedna od danas najpoznatijih ulica u Parizu. O istorijskim dogadjajima toga vremena mozhemo samo da naslutimo iz rechenice \zn ...ta dva zeta, za vreme carstva, nisu ustezali da primaju starca iz '93; to je josh moglo ic1i sa Bonapratom. Ali kad su se vratili Burboni, stari je pocheo da smeta gospodinu de Restou, a jos vishe bankaru.". Primec1uje se uticaj Napoleonovih osvajanja, kao i francuske revolucije. Iz datog citata vidimo povratak dinastije Burbona koji je okarakterisan materijalistichkim idealima u toj meri da se ljudi mere vishe novcem nego chinovima. Ljudi postaju sebichni, hladni bezosec1ajni, sve se vrti oko novca.
\end{p}

\begin{p}
\textbf{Odredi shiru i uzhu temu romana i, na osnovu toga, odredi tip romana.}\\
Uzhe teme romana su propast oca koji je bezuslovno voleo svoje c1erke i zhelja sirmoashnog studenta da uspe u viskom drushtvu, a shira je parisko drushtvo pochetkom 19. veka i njihova tezhnja ka moc1i i bogat{}stvu. Roman je slozhen, ima vishe tipova od kojih je najznachajniji tip karaktera, ali roman je i drushtvenog, ljubavnog kao i kriminalistichkog tipa.
\end{p}

\begin{p}
\textbf{Chicha Gorio - analiza lika.}\\
%101. strana
Zhan Zhoashen Gorio bio je obichan radnik u fabrici suvoga testa. On je bio chovek koji je imao sposobnosti da trguje sa zhitaricama i zbog toga je stekao mnogo novca. Bio je strpljiv, vredan preduzimljiv, istrajan i brz u izvrshavanju svojih planova. Nakon shto mu je zhena umrla posle 7 godina braka on ostaje sam sa dve c1erke kojima c1e posvetiti ostatak svog zhivota. Sa obzirom da je imao puno novca Gorio je mogao da ispuni svaku zhelju svojih c1erki. On je zadovoljavao svaku c1ud svoje dece, pritom ne troshec1i ni paru na sebe. Ovakvo ponashanje imac1e velike posledice u buduc1em zhivotu. Gorio je \textbf{monoman}, jedina stvar o kojoj on razmishlja su njegove c1erke.\\

Gorio se nakon udaje njegovih c1erki seli u pansion gospodje Voke da bi bio blizhe c1erkama. On prvobitno plac1a vec1u sumu za smeshtaj, a zatim sve manju i manju jer mu njegove c1erke iziskuju sve vishe novaca. Medjutim sve se zakomplikuje kada njegovi zetovi zabranjuju njihovim zhenama da vidjaju njihovog oca. Gorio se povlachi u sebe i kao da je izgubio svaku radost u zhivotu on postaje monotona mashina i polako kao da pada u hibernaciju. Na pochetku on uzhiva poshtovanje ostalih ukuc1ana ali vremenom gubi to poshtovanje i dobija nadimak chicha Gorio. Ovome doprinosi to shto mu c1erke tajno dolaze tako da svi \\pomishljaju da Gorio plac1a za zhensko drushtvo. Ovaj junak je potpuno zanemaren sve do trenutka kada Rastinjak upoznaje gospodjicu de Resto i obaveshtava ostale ukuc1ane da je to stvarno c1erka gospodina Goria. U tom trenutku ozhivljava lik Goria koji krec1e da bunca i koji se budi na pomen njegovih c1erki. Gorio c1e posle ovoga biti privrzhen Rastinjaku koji je ocharan lepotom prvo jedne, a zatim i druge c1erke chicha Goria, gospodjom Nizhensen. Jedina stvar do koje je Goriu stalo je srec1a njegovih c1erki i on je spreman da se odrekne svega shto ima da bi zadovoljio potrebe svojih razmazhenih c1erki kojima su muzhevi uskratili novac i lishili ih uzhivanja u braku.\\

Grofica de Resto i njena sestra baronica iskorish$ $c1avale su oca jer su im njihovi muzhevi uskratili novac na toaletu. Gorio je bio zaslepljen prekomernom ochinskom ljubavlju i prodavao je i poslednje stvari koje su mu ostale samo da bi zadovoljio potrebe svojih c1erki. Medjutim njima je uvek trebalo josh. Nakit, haljine, pa chak i stan za sastajanje sa ljubavnikom nisu bili dovoljni. Chak i kada je bio na samrti, Gorio odlazi do zelenasha da bi prodao escajg, poslednju stvar od vrednosti koju on poseduje.\\

Medjutim u trenutku kada ostaje bez novca i na samrti, Goriu ne dolazi ni jedna c1erka. On s{}hvata da se obistinilo ono chega se bojao poslednjih 10 godina ali i dalje se nada da ga c1erke nec1e ostaviti da umre sam. U takvom trenutku Gorio s{}hvata da novac ne mozhe kupiti ljubav. Kaje se zbog toga shto je razmazio c1erke i zhrtvovao se za njih u svakom trenutku. 
\end{p}

\begin{p}
\textbf{Ezhen de Rastinjak - analiza lika.}\\
Rastinjak potiche iz siromashne porodice sa juga Francuske i dolazi u Pariz prvenstveno kako bi se shkolovao ali i sa potajnom zheljom da postane chlan visokog drushtva.\zn Njegov otac, njegova majka, njegove dve sestre, njegova dva brata i jedna tetka cije se sve imanje sastojalo iz nekoliko penzija zhiveli su na malome dobru Rastinjakovih." Rastinjak dolazi iz velike radnichke porodice koja skromno ali lepo zhivi. On je inteligentan, moralan i shtedljiv, znao je vrednost novca.\\

%strana 53
Ezhenova tetka, gospodja de Marsijak poznavala je i najuglednije predstavnike aristokratije i uspela je da mu obezbedi ulaznicu u visoko drushtvo, pismo viskontesi de Bosean. Rastinjak je spremno dochekao poziv na bal viskontese i uspeo je da se pokazhe u tom drushtvu kao i da medju svim tim prelepim zhenama odabere groficu de Resto, chija ga je lepota naprosto ocharala. On odluchuje da poseti gospodju de Resto ali biva razocharan spoznajom da njeno srce pripada njenom ljubavniku Maksimu de Traju. Medjutim ne uspeva da se dopadne ni gospodinu de Resto. Zbunjeni Rastinjak ne znajuc1i vezu \zn chicha Goria" sa groficom pominje Goria i dozhivljava strogo negodovanje kod porodice de Resto i zatvaraju mu se vrata ka ovoj kuc1i. Iskusivshi neprijatnost, on odlazi pravo kod gospodje de Bosean gde saznaje neverovatnu istinu, grofica de Resto je c1erka chicha Goria. Ovo saznanje deli sa ostalim ukuc1anima i staje na stranu Goria koji se budi na pomen njegovih c1erki.\\ 

Viskontesa objashnava Rastinjaku da nec1e uspeti ako bude bio poshten. \zn ...postupajte sa tim svetom kako zashluzhuje. Udarajte bez sazhaljenja i svet c1e vas se bojati." Ona mu govori da ne sme izrazhavati osec1anja, da ne sme verovati drugome, da mora hladno i sebichno postupati sa ljudima. Viskontesa de Bosean uzima Rastinjaka pod svoje okrilje i daje mu za pravo da koristi njeno ime kako bi dospeo u slojeve visokog drushtva. Ona mu govori i da je gospodjica Nizhensen prava prilika za njega i Rastinjak odluchuje da iskoristi tu shansu i priblizhi se baronici.\\

S{}hvativshi iz susreta sa gospodjom de Resto da svaki susret u viskokom drushtvu iziskuje nezanemarljivu kolichinu novca i vremena Rastinjak pishe pismo porodici u kojem trazhi da mu poshalju josh novca. Od ovog trenutka Rastinjakov karakter se polako menja. Skroman i disciplinovan dechak znajuc1i u kakvoj bedi njegova porodica zhivi usudjuje se da trazhi novac od majke i sestara znajuc1i da ga one neizmerno vole i da c1e se odrec1i svega samo da bi mu taj novac poslale. Kada je primio pisma u isti mah je pretrnuo od radosti i zadrhtao od uzhasa. Rastinjak je toliko zheleo da se infiltrira u visoko drushtvo da mu je njegova sebichnost dozvoljavala igru sa osec1anjima njegovih blizhnjih. Pisac stavlja Rastinjaka pred razna iskushenja i pomno prati razvoj njegovog kakaktera kroz brojne prepreke.\\

Nakon shto se Ezhen odluchio da pokusha kod gospodje Nizhensen, Votren za njega ima drugih planova. Uvidevshi sa kakvom hrabrosh$ $c1u se odlikuje Rastinjakm, Votren mu izlazhe svoj mrachni plan. Sve shto trazhi od Rastinjaka je da se ozheni gdjicom Tajfer, chijeg c1e brata njegovi ljudi ubiti i ona c1e postati jedna od najbogatijih gradjanki Pariza. Jedino shto je Ezhen trebao da uchini je se ozheni gdjicom Tajfer, a potom da njenim velikim mirazom isplati deo Votrenu. Takodje kao i viskontesa de Bosean, Votren je Rastinjaku predochio osobine drushtva u kojem zhivimo. Objasnio mu je da kao sudija nec1e daleko dogurati i da je jedini nachin da postane neshto u ovom drushtvu da pochini ovakav gnusan zlochin. Medjutim ovo je za Ezhena bio preveliki zalogaj. Iako se u kritichnim situacijama dosta kolebao, Ezhena je ipak odrzhavala ljubav baronice Nizhensen i uspeo je da se otrgne zloj nameri Votrena. \\

Tokom cele priche Rastinjak je tu da iz lichog ugla posmatra propadanje Goria tokom vremena. Mladic1 iskreno saosec1a za starcem i jedini razume i podrzhava bezuslovnu ochinsku ljubav. Kada Gorio prodaje i poslednju stvar da bi pomogao c1erki, Ezhen ostaje uz njega i prodaje sve shto ima samo da bi pomogao ostavljenom starcu na samrti. C1erke chicha Goria ne dolaze chak ni na njegovu sahranu vec1 samo shalju prazne kochije shto bi moglo simbolisati prazno srce. Dva studenta posmatraju stravichan prizor - oca koje su c1erke ostavile da umre u samoc1i. U tom trenutku raspleta studenti shvataju izopachenost drustva u kome zhive. Poslednja Ratinjakova rechenica \zn Sad je na nas dvoje red." ukazuje na to da je on svestan kako se treba odnositi prema tom sebichnom i oholom drushtvu i spreman je da takorec1i uchini sve shto je u njegovoj moc1i da postane deo visokog pariskog drushtva.
Rastinjak se znachajno promeio od pochetka, njegov moral je izbledeo i njegov sistem vrednosti je nepovratno izmenjen. On s{}hvata da poshten chovek ne mozhe uspeti u takvom okruzhenju i da su Votren i viskontesa od pochetka bili u pravu.
\end{p}

\begin{p}
\textbf{Balzakov realizam.}\\
Balzak se u svojim delima znachajno bavio ljudskim lichnostima svog doba. Likove je sagledavao iz vishe uglova. Svi njegovi likovi su izrazito slozheni, sa razlichitim motivima, idejama, zheljama, pogledima na svet. U samom \zn Chicha Goriu“ prikazao je drushtvena pravila i ideale u tadashnjem Parizu, za koje se ne mozhe rec1i da se toliko razlikuju od danashnjih.
\end{p}

\end{document}