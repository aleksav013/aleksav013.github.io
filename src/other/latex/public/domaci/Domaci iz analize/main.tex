\documentclass[11pt]{article}
\usepackage[a4paper,left=15mm,top=10mm,bottom=5mm,right=15mm]{geometry}
\usepackage[OT2]{fontenc}
\pagenumbering{gobble}
\usepackage{amsfonts}
\usepackage{amsthm,amsmath}
\newcommand\tsd{\theoremstyle{definition}}
\newcommand\tsr{\theoremstyle{remark}}

\usepackage{standalone}
\usepackage{pgf,tikz,pgfplots}
\usetikzlibrary{arrows}
\pgfplotsset{compat=1.16}

\usepackage{multicol}
\usepackage{float}
\addtolength{\jot}{-1mm}
\usepackage{mathtools}

\tsd\newtheorem{zad}{Zadatak}
\newcommand\eng{\fontencoding{OT1}\fontfamily{\rmdefault}\selectfont}
\newcommand\srb{\fontencoding{OT2}\fontfamily{\rmdefault}\selectfont}
\newcommand{\D}{\displaystyle}

\title{\bf{Domac1i zadatak iz analize}}
\author{\Large Aleksa Vuchkovic1, 2c}
\date{}

\begin{document}
\maketitle
\large

\begin{zad} Skicirati grafike sledec1ih funkcija:\\

\begin{multicols}{2}
a) $f(x)=5^{3-x}$
\begin{figure}[H]
    \includestandalone[scale=0.5]{1zad/1a}
    \label{fig:2a}
\end{figure}
\columnbreak
b) $f(x)=3\cdot 5^{-|x|}$
\begin{figure}[H]
    \includestandalone[scale=0.7]{1zad/1b}
    \label{fig:2b}
\end{figure}
\end{multicols}

\begin{multicols}{2}
v) $f(x)=3-5^{-|x|}$
\begin{figure}[H]
    \includestandalone[scale=0.7]{1zad/1v}
    \label{fig:2v}
\end{figure}
\columnbreak
g) $f(x)=2-|5-5^{-x}|$
\begin{figure}[H]
    \includestandalone[scale=0.6]{1zad/1g}
    \label{fig:2g}
\end{figure}
\end{multicols}

\begin{multicols}{2}
d) $f(x)=max\{5^{-x},5^x\}$
\begin{figure}[H]
    \includestandalone[scale=0.6]{1zad/1d}
    \label{fig:2d}
\end{figure}
\columnbreak
dj) $f(x)=min\{5^{-x},5^x\}$
\begin{figure}[H]
    \includestandalone[scale=0.9]{1zad/1dj}
    \label{fig:2dj}
\end{figure}
\end{multicols}

\end{zad}

\begin{zad} Odrediti domen i skup vrednosti funkcije:\\[1mm]
a) $f(x)=3^{\sqrt{1-x^2}}\Rightarrow x\in [-1,1]\Rightarrow f(x)\in [1,3]$\\
b) $f(x)=2^{\sqrt{x^2-1}}\Rightarrow x\in (-\infty,-1]\cup [1,+\infty)\Rightarrow f(x)\in [1,+\infty)$\\
v) $f(x)=\left (\frac{1}{3}\right )^{\sin{x}}\Rightarrow x\in\mathbb{R}\Rightarrow f(x)\in [\frac{1}{3},3]$\\[0.5mm]
g) $f(x)=\sqrt{0.5^x-8}=\sqrt{2^{-x}-2^3}\Rightarrow 2^{-x}-2^3\leq 0 \Rightarrow x\leq -3\Rightarrow x\in(-\infty,-3]\Rightarrow f(x)\in [0,+\infty)$\\
\end{zad}

\newpage

\begin{zad} Reshiti jednachine:\\

\begin{multicols}{2}
\noindent
\begin{align*}
\text{a) } \left( \frac{1}{2}\right)^{1-x}&=\sqrt 2\\
\left( \frac{1}{2}\right)^{1-x}&=\left( \frac{1}{2}\right)^{-\frac{1}{2}}\\
1-x&= -\frac{1}{2}\\
\Aboxed{x&=\frac{3}{2}}\cdot\\
\end{align*}
\begin{align*}
\text{b) } 15^x&=5\cdot 3^x\\
\not{3^x} \cdot 5^x&=5\cdot \not{3^x}\\
\Aboxed{x&=1}.
\end{align*}
\begin{align*}
\text{v) } 3\cdot 2^{2x-1}&=6^x\\
\frac{3}{2}\cdot 4^{x}&=6^x\\
\frac{3}{2}&=\left(\frac{6}{4}\right)^x\\
\frac{3}{2}&=\left(\frac{3}{2}\right)^x\\
\Aboxed{x&=1}.
\end{align*}
\begin{align*}
\text{g) } 36^x-42\cdot 6^x+216&=0\\
6^{2x}-42\cdot 6^x+216&=0\\
6^x&=\frac{42\pm\sqrt{42^2-4\cdot 216}}{2}\\
6^x&=\frac{42\pm 30}{2}\\
6^x=36 &\lor 6^x=6\\
\Aboxed{x&\in\{1,2\}}.
\end{align*}
\begin{align*}
\text{d )} \sqrt{\left(5\sqrt 2 +7\right)^x}&+\sqrt{\left(5\sqrt 2 -7\right)^x}=198\\
t&=\sqrt{\left(5\sqrt 2 +7\right)^x}\\
t+\frac{1}{t}&=198\\
t^2-198t+1&=0\\
t&=\frac{198\pm\sqrt{198^2-4}}{2}\\
t&=99\pm 70\sqrt{2}\\
t&=99\pm 70\sqrt{2}=\left(5\sqrt 2 +7\right)^\frac{x}{2}\\
99+70\sqrt{2}&=(5\sqrt{2}+7)^2,\\
99-70\sqrt{2}=(5\sqrt{2}-7)^2&=\frac{1}{(5\sqrt{2}+7)^2}=(5\sqrt{2}+7)^{-2},\\
\frac{x}{2}&=\pm2\\
\Aboxed{x&=\pm4}.
\end{align*}
\begin{align*}
\text{dj) } 4\cdot 2^{2x}=18\cdot 3^{2x}+6^x/:2^x\\
4=18\cdot \left(\frac{3}{2}\right)^{2x}+\left(\frac{3}{2}\right)^x\\
t=\left(\frac{3}{2}\right)^x\\
0=18t^2+t-4\\
t=\frac{-1\pm \sqrt{1+4\cdot 18\cdot 4}}{36}\\
t=-\frac{1}{2} \lor t=\frac{4}{9}\\
\left(\frac{3}{2}\right)^x=\frac{4}{9} \lor \left(\frac{3}{2}\right)^x=-\frac{1}{2}\\
x=-2 \lor \bot \\
\Aboxed{x\in\{-2\}}.
\end{align*}
\begin{align*}
\text{e) } 9\cdot (9^x+9^{-x})-3\cdot (3^x+3^{-x})=72\\
t=3^x\\
9\cdot (t^2+\frac{1}{t^2})-3\cdot (t+\frac{1}{t})=72\\
y=t+\frac{1}{t}\\
y^2=t^2+\frac{1}{t^2}+2\\
9\cdot(y^2-2)-3\cdot y=72\\
3\cdot y^2-6-3\cdot y=72\\
3\cdot y^2-y-30=0\\
y_{1,2}=\frac{1\pm\sqrt{1+30\cdot3\cdot4}}{6}\\
y_1=-3 \lor y_2=\frac{10}{3}\\
\frac{10}{3}=t+\frac{1}{t}\\
t^2-\frac{10}{3}\cdot t +1=0\\
t=\frac{\frac{10}{3}\pm\sqrt{\frac{100}{9}-4}}{2}\\
t_1=3 \lor t_2=\frac{1}{3}\\
\Aboxed{x_{1,2}=\pm1}\\
-3=t+\frac{1}{t}\\
t^2+3\cdot t+1=0\\
t=\frac{-3\pm\sqrt{9-4}}{2}\\
t=\frac{-3\pm\sqrt{5}}{2}\\
3^x=\frac{-3\pm\sqrt{5}}{2}.\\
\text{Drugi deo nema reshenja,}\\
\text{jer je desna strana jednachine manja od 0}.\\
\text{Stoga jedina reshenja su:}\\
\Aboxed{x\in\{1,2\}}.
\end{align*}
\begin{align*}
\text{zh )} \left( \left(\sqrt[6]{81}\right)^{\frac{x}{5}-\sqrt{x}}\right)^{\frac{x}{5}+\sqrt{x}}=3^\frac{9}{5}\\
\left( \left(3^\frac{4}{6}\right)^{\frac{x}{5}-\sqrt{x}}\right)^{\frac{x}{5}+\sqrt{x}}=3^\frac{9}{5}\\
\left(\frac{4}{6}\right)\cdot\left(\frac{x}{5}-\sqrt{x}\right)\cdot\left(\frac{x}{5}+\sqrt{x}\right)=\frac{9}{5}\\
\left(\frac{2}{3}\right)\cdot\left(\frac{x^2}{25}-x\right)\cdot=\frac{9}{5}\\
x^2-25x-\frac{135}{2}=0\\
x=\frac{25\pm\sqrt{625+\frac{135}{2}\cdot4}}{2}\\
\Aboxed{x=\frac{25+\sqrt{895}}{2}}\text{ jer je }x>0.
\end{align*}
\begin{align*}
\text{z) } 2^x+3^x=5^x/:3^x\\
\left(\frac{2}{3}\right)^x+1=\left(\frac{5}{3}\right)^x\\
\text{Leva strana je opadajuc1a, a desna rastuc1a,}\\
\text{Stoga je jedino moguc1e reshenje:}\\
\Aboxed{x=1}.
\end{align*}
\begin{align*}
\text{i) } 10^x+11^x+12^x=13^x+14^x/:12^x\\
\left(\frac{5}{6}\right)^x+\left(\frac{11}{12}\right)^x+1=\left(\frac{13}{12}\right)^x+\left(\frac{7}{6}\right)^x\\
\text{Leva strana je opadajuc1a, a desna rastuc1a,}\\
\text{Stoga je jedino moguc1e reshenje:}\\
\Aboxed{x=2}.
\end{align*}
\end{multicols}
\end{zad}



\begin{zad} Reshiti nejednachine:\\[-5mm]
\begin{multicols}{2}
\noindent
\begin{align*}
\text{a) } \left( \frac{5}{4}\right)^{1-x}<(0,64)^{2(1+\sqrt x)}\\
\left( \frac{5}{4}\right)^{1-x}<\left(\frac{4}{5}\right)^{4(1+\sqrt x)}\\
\left( \frac{5}{4}\right)^{1-x}<\frac{5}{4}^{-4(1+\sqrt x)}\\
1-x<-4\cdot(1+\sqrt{x})\\
0<x-4\cdot\sqrt{x}-5\\
(\sqrt{x}-5)\cdot(\sqrt{x}+1)>0\\
\sqrt{x}\geq0 \land \sqrt{x}>5\Rightarrow x>25\\
\Aboxed{x\in(25,+\infty)}.
\end{align*}
\begin{align*}
\text{b) } 10^{4x^2-2x-2}\geq 100^{x-1,5}\\
10^{4x^2-2x-2}\geq 10^{2x-3}\\
4\cdot x^2-2\cdot x-2\geq 2\cdot x-3\\
4\cdot x^2-4\cdot x+1\geq0\\
(2\cdot x-1)^2\geq0\\
\Aboxed{x\in\mathbb{R}}.
\end{align*}
\begin{align*}
&\text{v) } \frac{3^x-9}{x^2-5x+6}>0\\
&\frac{3^x-3^2}{(x-3)\cdot(x-2)}>0\\
&i)x<2\Rightarrow 3^x-3^2<0, (x-3)\cdot(x-2)>0\Rightarrow\bot\\
&ii)2\leq x\leq3\Rightarrow 3^x-3^2>0, (x-3)\cdot(x-2)<0\Rightarrow\bot\\
&iii)x>3\Rightarrow 3^x-3^2>0, (x-3)\cdot(x-2)>0\Rightarrow x>3\\
&\Aboxed{x\in(3,+\infty)}.
\end{align*}
\begin{align*}
\text{g) } 3^{x-1}+3^{x-2}-3^{x-4}<315\\
3^x\left(\frac{1}{3}+\frac{1}{9}-\frac{1}{81}\right)<315\\
3^x\left(\frac{35}{81}\right)<315\\
3^x<81\cdot9\\
3^x<3^6\\
\Aboxed{x\in(-\infty,6)}.
\end{align*}
\begin{align*}
\text{d) }  \sqrt{\left(5+2\sqrt 6\right)^x}+\left(\sqrt 3 -\sqrt 2\right)^x\leq 10\\
t=\left(\sqrt 3 -\sqrt 2\right)^x\\
\sqrt{\frac{1}{t^2}}+t\leq10\\
t+\frac{1}{t}\leq 10\\
t^2-10\cdot t +1\leq 0\\
t=5\pm2\cdot\sqrt{6}\\
t\in[(\sqrt 3 -\sqrt 2)^{-2},(\sqrt 3 -\sqrt 2)^2]\\
\Aboxed{x\in\{-2,2\}}.
\end{align*}
\begin{align*}
&\text{dj) } 3^{x+2}\geq 3^{2x+5}-x\\
&x\geq 3^{2x+5}-3^{x+2}\\
&i)x<-3\Rightarrow 3^{2x+5}-3^{x+2}>-1\Rightarrow\bot\\
&ii)-3\leq x<0\Rightarrow 3^{2x+5}-3^{x+2}\geq0\Rightarrow\bot\\
&iii)0\leq x\Rightarrow f(x)=3^{2x+5}-3^{x+2} \\
&f(x)\text{ raste eksponencijalno za razliku od }x\\
&\text{Stoga ova nejednakost nema reshenja.}\\[-5mm]
\end{align*}
\begin{align*}
\text{e) } 2^x+3^x+12<5^x/:5^x&\\
1-\frac{2}{5}^x-\frac{3}{5}^x-\frac{12}{5^x}>0&\\
f(x)=1-\left(\frac{2}{5}\right)^x-\left(\frac{3}{5}\right)^x-\frac{12}{5^x}\text{ je }\nearrow&\\
\text{za }x=2\text{ } f(x)=0\Rightarrow x>2&\\
\Aboxed{x\in(2,+\infty)}.
\end{align*}
\end{multicols}
\end{zad}

\end{document}