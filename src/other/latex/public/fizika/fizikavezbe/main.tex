\documentclass[11pt]{article}
\usepackage[a4paper,left=20mm,top=10mm]{geometry}
\usepackage[OT2]{fontenc}
\pagenumbering{gobble}
\usepackage{amsfonts}
\usepackage{amsthm}
\newcommand\tsd{\theoremstyle{definition}}
\newcommand\tsr{\theoremstyle{remark}}

\tsd\newtheorem{vez}{Vezhba}
\newcommand\eng{\fontencoding{OT1}\fontfamily{\rmdefault}\selectfont}
\newcommand\srb{\fontencoding{OT2}\fontfamily{\rmdefault}\selectfont}

\title{\bf{Vezhbe iz fizike}}
\author{\Large Aleksa Vuchkovic1, 2c}
\date{}

\begin{document}
\maketitle
\large
\begin{vez} \textbf{Protok fluida i pritisak}\\
U prvom delu ove simulacije uochavamo zavisnost pritiska od gustine, dubine i visine fluida u kojem se meri. U drugom delu mozhemo meriti brzinu protoka fluida kroz cev i videti da ona zavisi od poprechnog preseka cevi, a ne zavisi od gustine fluida. Takodje, uochavamo da kada postoji trenje brzina fluida je manja blizhe cevi, a vec1a na sredini upravo zbog tog trenja. U trec1em delu simulacije imamo isticanje technosti kroz mali otvor na visini i gde mozhemo uochiti zavisnot brzine i dometa fluida od visine vode u sudu, kao i visine suda u odnosu na tlo.
\end{vez}
\begin{vez} \textbf{Potisak}\\
Primec1ujemo da sila potiska zavisi od gustine fluida, kao i od potopljene zapremine tela u fluid. Na primer dva tela iste gustine nec1e imati isti potisak u ravnotezhnom stanju zbog toga shto je jedno telo vishe potopljeno nego drugo zbog vec1e mase. Ali c1e zato dva tela sa istom potopljenom zapreminom u isti fluid imati iste sile potiska.
\end{vez}
\begin{vez} \textbf{Sudari u mehanici}\\
Prikaz zakona o odrzhanju implusa, kao i zakona o odrzhanju energije, ali samo u apsolutno elastichnim sudarima. Takodje, mozhe se uochiti zavisnost raspodele implusa od mase.
\end{vez}
\begin{vez} \textbf{Mase i opruge}\\
U ovoj simulaciji pokazana je zavisnost vremena oscilacije od konstante elastichnosti, mase tega, pochetne sile. Dati su merachi energije koji prikazuju vrednost energija, kao i rad utroshen na toplotu. Osim toga dat je i prikaz vektora brzine i sila koji u svakom trenutku prikazuju intenzitet svih sila sistema.
\end{vez}
\begin{vez} \textbf{Mesechev modul}\\
Zanimljiva igrica koja prikazuje delovanje sile potiska u cilju ublazhavanja pada, u ovom sluchaju sletanja na Mesec. Igrach ima za cilj da sleti pri brzini manjoj od $2\frac{m}{s}$, tj. "meko" sletanje da bi sachuvao letelicu.
\end{vez}
\end{document}