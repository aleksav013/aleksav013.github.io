\documentclass[11pt]{article}
\usepackage[ a4paper,total={170mm,257mm},left=20mm,top=10mm]{geometry}
\usepackage[OT2]{fontenc}
\pagenumbering{gobble}
\usepackage[version=4]{mhchem}
\usepackage{amsthm}
\newcommand\tsd{\theoremstyle{definition}}
\newcommand\tsr{\theoremstyle{remark}}

\tsd\newtheorem{og}{Ogled}
\newtheorem*{zak}{Zakljuchak}
\newcommand\eng{\fontencoding{OT1}\fontfamily{\rmdefault}\selectfont}
\newcommand\srb{\fontencoding{OT2}\fontfamily{\rmdefault}\selectfont}
\newcommand\ii{\eng{(II)}\srb{}}
\newcommand\iii{\eng{(III)}\srb{}}

\title{\bf{Labaratorijske vezhbe iz hemije}}
\author{\Large Aleksa Vuchkovic1 2c}
\date{}

\begin{document}
\maketitle
\large
\begin{og} Dobijanje gvozhdje\ii-hidroksida:\\\\
U natrijum-hidroksid dodati gvozhdje\ii-sulfat.\\
Izdvaja se zamuc1enje(talog) maslinasto zelene boje, a to je \eng{\ce{Fe(OH)2}}.\\
\eng{\ce{2NaOH +  FeSO4 -> Fe(OH)2 v + Na2SO4}}
\end{og}
\begin{og} Dobijanje gvozhdje\iii-hidroksida:\\\\
U gvozhdje\iii-hlorid dodati natrijum-hidroksid.\\
Izdvaja se zamuc1enje(talog) mrke boje, a to je \eng{\ce{Fe(OH)3}}.\\
\eng{\ce{FeCl3 + 3NaOH -> Fe(OH)3 v + 3NaCl}}
\end{og}
\begin{og} Gradjenje kompleksa gvozhdja "berlinsko plavo":\\\\
U gvozhdje\iii-hlorid dodati kalijum-heksacijanoferat\ii.\\
Boja prelazi iz zhuto/narand2aste u tamno plavu, tkz. berlinsko plavu.\\
\eng{\ce{4FeCl3 + 3K4[Fe(CN)6] -> Fe4[Fe(CN)6]3 + 12KCl}}
\end{og}
\begin{og} Gradjenje kompleksa gvozhdja "veshtchka krv":\\
U gvozhdje\iii-hlorid dodati kalijum-tiocijanat.\\
Boja postaje tamno crvena kao krv, po chemu je ovo jedinjenje dobilo naziv veshtachka krv.\\
\eng{\ce{FeCl3 + 6KSCN ->  K3[Fe(SCN)6] + 3KCl}}
\end{og}
\begin{og} Dobijanje kobalt\ii-hidroksida:\\\\
U kobalt\ii-hlorid dodati natrijum-hidroksid.\\
Uochavamo promenu boje iz blago crvene u plavu.\\
\eng{\ce{CoCl2 + 2NaOH -> Co(OH)2 + 2NaCl}}
\end{og}
\begin{og} Dobijanje nikal\ii-hidroksida:\\\\
U vodeni rastvor nikal\ii-sulfata dodati natrijum-hidroksid.\\
Uochavamo nastajanje voluminoznog taloga zelene boje.\\
\eng{\ce{NiSO4 + 2NaOH -> Ni(OH)2 v + Na2SO4}}
\end{og}
\begin{og} Gradjenje kompleksa nikla:\\\\
U nikal\ii-hlorid dodati amonijak.\\
Pre pochetka reakcije, imamo zelenu boju rastvora nikal\ii-hlorida. U zavisnosti od dodate kolichine amonijaka pojachava se plava boja.\\
\eng{\ce{NiCl2 + NH3 + 2H2O <=> [Ni(NH3)6]Cl2 }}
\end{og}
\end{document}