\documentclass{article}
\usepackage[ a4paper,total={170mm,257mm},left=20mm,top=0mm,bottom=10mm]{geometry}
\usepackage[OT2]{fontenc}
\usepackage{multicol}
\usepackage{amssymb}
\usepackage{amsmath}
\newcommand\texteng{\fontencoding{OT1}\fontfamily{\rmdefault}\selectfont}
\pagenumbering{gobble}
\title{\textbf{\huge Geometrijske nejednakosti}}
\date{}
\begin{document}
\setlength{\columnsep}{1.2cm}
\maketitle
\begin{center}
\vspace*{-1.5cm}
 \large\textit{\textbf{Autor:} Aleksa Vuchkovic1, uchenik 1. razreda Matematichke gimnazije}\\
    \textit{\textbf{Mentor:} Branko Grbic1}\\
    \normalsize\textit{Regionalni centar za talente "Mihajlo Pupin", Dimitrija Tucovic1a 2, Panchevo}
\end{center}
\begin{multicols}{2}
 \noindent
\section*{\underline{Uvod}}

\large Shiroku klasu nejednakosti koje se srec1u u primenama chine \text geometrijske nejednakosti. Pod geometrijskom nejednakosh$ $c1u se najchesh$ $c1e podrazumeva ona nejednakost koja vazhi za elemente trougla ili neke
druge geometrijske \text figure (chetvrougla, kupe, valjka, lopte, itd.). U shirem smislu, {geometrijska} nejednakost je svaka {nejednakost} koja se odnosi na geometrijski crtezh.
\section*{\underline{Cilj}}
Cilj istrazhivachkog rada je obrada i \text{primena} elementarnih nejednakosti u
{nastavi} matematike u osnovnoj i \text srednjoj shkoli, kao i pripremi za
\text takmichenja. Rad obuhvata sintezu teorije sa \text primenom, što je standard savremene literature.
\section*{\underline{Metoda rada}}
Metode istrazhivanja su \text prikupljanje lit{erature}, kao i njihovo tumachenje i \text{reshavanje}.
Trazhena je korelacija izmedju teorijskih postavki i \text praktichne
primene, detaljno je prikazan \text proces dolazhenja do reshenja. U radu je
korish$ $c1ena kombinovana metoda \text istrazhivanja, prouchavana je
literatura gde je raznovrsna primena \text{nejednakosti}.
\noindent
\section*{\underline{Rezultati istrazhivanja}}
Predmet obrade ovog \text nauchnoistrazhivachkog rada su geometrijske nejednakosti u ravni, sa posebnim akcentom na \text nejednakosti u trouglu. Pored elementarnih planimetrijskih nejednakosti dat je osvrt na {Ptolomejevu nejednakost} $$AB\cdot CD+AD\cdot BC\geq AC\cdot BD$$ koja vazhi i u prostoru i chija je \text{primen}ljivost i efektivnost u reshavanju takmicharskih zadataka pokazana u ovom radu. Osim toga pokazana je i Ojlerova nejednakost $$R\geq 2r$$ koja je i dalje jedno od najvec1ih otkric1a u ovoj oblasti matematike u protekla dva i po veka. Iako je \text jednostavna, Ojlerova {nejednakost} nije ni koji nachin trivijalna i doprinosi razumevanju odnosa dva vazhna {aspekta} trougla. Slikovno su  {predstavljene} i nejednakosti izmedju brojevnih sredina, koje na neki nachin pokazuju povezanost izmedju same \text algebre i geometrije. Osim prethodno \text pomenutih stvari,  nachinjen je izbor \text najbitnijih stvari na ovu temu po \text autorovom mishljenju sa kojima c1e se chitalac susreti u predstojec1em radu.
\section*{\underline{Zakljuchak}}
Nejednakosti se mogu koristiti
kao dobar priruchnik matemeticharima, fizicharima, inzhenjerima,
mehanicharima, statisticharima, ekonomistima.
\text Primena nejednakosti je zastupljena u \text matematichkoj analizi, geometriji, teoriji verovatnoc1e, matemetichkoj statistici, \text matematichkoj obradi podataka, linearnom i \text dinamichkom programiranju kao i u \text teorijskoj i primenjenoj matematici.
\section*{\underline{Literatura}}
[1] Z. Kadelburg, D. Djukic1, M. Lukic1, I. Matic1, \textit{Nejednakosti}, 2. dopunjeno \text izdanje,  DMS, Beograd 2014.\\ {}
[2] M. Mitrovic1, S. Ognjanovic1, M. Veljkovic1, Lj. Petrovic1, N. Lazarevic1,\\ \textit{Geometrija za prvi razred Matematichke gimnazije}, 4. izdanje, Krug, Beograd 2013.\\{}
[3] Z. Cvetkovski,
\textit{\texteng{Inequalities, Theorems, Techniques and Selected Problems}}\\{} 
[4] {\texteng{O. Bottema}}, R. Zh. Djordjevic1, R. R. Janic1, D. S. Mitrinovic1, P. M. Vasic1,\\
\texteng{\textit{Geometric Inequalities}, Wolters-Noordhoff Publishing, Groningen, Netherlands 1969.}

 \end{multicols}
  \pagebreak

\end{document}
