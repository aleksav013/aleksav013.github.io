\documentclass{beamer}
\mode<presentation>
\usetheme{Warsaw}

\definecolor{mygreen}{rgb}{.125,.5,.25}
\usecolortheme[named=mygreen]{structure}

\definecolor{lightgr}{rgb}{0.7 0.7 0.7}
\makeatletter
\addtobeamertemplate{footline}{%
  \color{lightgr}% to color the progressbar
  \hspace*{-\beamer@leftmargin}%
  \rule{\beamer@leftmargin}{2pt}%
  \rlap{\rule{\dimexpr
      \beamer@startpageofframe\dimexpr
      \beamer@rightmargin+\textwidth\relax/\beamer@endpageofdocument}{1.5pt}}
  % next 'empty' line is mandatory!

  \vspace{0\baselineskip}
  {}
}

\usepackage[OT2]{fontenc}
\usepackage{amsfonts,amssymb,amsthm,amsmath}

\def\zn{,\kern-0.09em,}
\setlength\intextsep{0pt}

\newcommand\eng{\fontencoding{OT1}\fontfamily{\rmdefault}\selectfont}
\newcommand\srb{\fontencoding{OT2}\fontfamily{\rmdefault}\selectfont}

\title{\textbf{Naslov}}
\author{Aleksa Vuchkovic1 3c}
\institute{Matematichka gimnazija, Beograd}
\date{}


\begin{document}
\maketitle

\begin{frame}{Sadrzhaj}
\tableofcontents
\end{frame}

\section{Uvod}
\subsection{Lopta1}
\subsection{Lopta2}
\begin{frame}{Uvod i dalje}
\begin{block}{Lmao ovo je blok}
Lopta je geometrijsko telo ogranicheno sferom. Chine je sve tachke koje su manje ili jednako udaljene $r$ od centra.
\end{block}
\onslide <2-3>
Lopta\\
\onslide <3>
\Large$P=4r^2\pi$\\
\Large$V=\frac{4}{3}r^3\pi$
\end{frame}

\section{}
\begin{frame}
\centering\LARGE HVALA NA PAZHNJI!
\end{frame}
\end{document}