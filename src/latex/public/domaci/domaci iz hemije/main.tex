\documentclass[11pt]{article}
\usepackage[ a4paper,left=20mm,top=10mm,right=10mm]{geometry}
\usepackage[OT2]{fontenc}
\pagenumbering{gobble}
\usepackage[version=4]{mhchem}
\usepackage{chemfig}
\usepackage{amsthm}
\newcommand\tsd{\theoremstyle{definition}}
\newcommand\tsr{\theoremstyle{remark}}
\setlength{\tabcolsep}{17pt}

\tsd\newtheorem{zad}{Zadatak}
\newcommand\eng{\fontencoding{OT1}\fontfamily{\rmdefault}\selectfont}
\newcommand\srb{\fontencoding{OT2}\fontfamily{\rmdefault}\selectfont}
\newcommand\ii{\eng{(II)}\srb{}}
\newcommand\iii{\eng{(III)}\srb{}}

\title{\bf{Domac1i zadatak, Uvod u organsku hemiju}}
\author{\Large Aleksa Vuchkovic1 2c}
\date{}

\begin{document}
\maketitle
\large
\begin{zad}a)\\
\begin{center}
    \chemfig{CH_3-CH(-[2]CH_3)-CH_2-CH_3}
\end{center}
b) \eng\ce{C5H12}
\end{zad}
\begin{zad}\leavevmode\\[-5mm]
\begin{table}[h]
\begin{tabular}{ll}
$V(C_xH_y)=1.12$\eng{dm}$^3$ & $M(C_xH_y)=\cfrac{V_mm}{V}=44\frac{g}{mol}$ \\
$m(C_xH_y)=2.2$\eng{g}     & $xM(C)+yM(H)=M(C_xH_y)$     \\[1mm]
$81,82\%\, \eng{C}$  & $\cfrac{xM(C)}{xM(C)+yM(H)}=0,8182$       \\[3mm]
$18,18\%\, \eng{H}$  & $xM(C)=M(C_xH_y)\cdot 0,8182=36\frac{g}{mol}$   \\
& $x=3\Rightarrow y=8$
\end{tabular}\\[5mm]
Molekulska, kao i empirijska formula su $C_3H_8$, zato shto je NZD(3,8)$=$1.
\end{table}

\end{zad}
\end{document}