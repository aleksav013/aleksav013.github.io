\documentclass[11pt]{article}
\usepackage[a4paper,left=20mm,top=10mm]{geometry}
\usepackage[OT2]{fontenc}
\pagenumbering{gobble}
\usepackage{amsfonts}
\usepackage{amsthm}
\usepackage{amsmath}
\newcommand\tsd{\theoremstyle{definition}}
\newcommand\tsr{\theoremstyle{remark}}

\newcommand\eng{\fontencoding{OT1}\fontfamily{\rmdefault}\selectfont}
\newcommand\srb{\fontencoding{OT2}\fontfamily{\rmdefault}\selectfont}

\title{\bf{Kondenzatori}}
\author{\Large Aleksa Vuchkovic1, 2c}
\date{}

\begin{document}
\maketitle
\Large\noindent
Kapacitivnost kondenzatora zavisi od rastojanja i povrshine plocha. Kapacitivnost je direktno srazmerna povrshini plocha, a obrnuto srazmerna rastojanju izmedju njih. Shto je vec1i napon, vec1e je naelektrisanje kao i energija koju sijalica mozhe da utroshi u cilju proizvodnje elektrichne energije. Kada ne postoji baterija ili elektromotor u kolu, napunjeni kondenzator se ponasha kao izvor enercije tj. pravi razliku potencijala. Bitno je napomenuti da naelekrisanje, kao i napon izmedju plocha zavise od jachine elekrichnog polja, pa ako \text menjamo napon izmedju plocha mi ne utichemo na kapacitivnost tog kondenzatora. Prethodna tvrdnja mozhe se izvesti direktno iz sledec1e formule:
$$C=\frac{Q}{U}=\frac{\varepsilon_0\cdot\varepsilon_r\cdot E\cdot S}{E\cdot d}=\frac{\varepsilon_0\cdot\varepsilon_r\cdot S}{d}\cdot$$

\end{document}